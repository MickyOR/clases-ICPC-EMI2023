\documentclass[10pt]{beamer}
\usepackage[utf8]{inputenc}
\usepackage[T1]{fontenc}
\usetheme{metropolis}
\usepackage{booktabs}
\usepackage[scale=2]{ccicons}
\usepackage{pgfplots}
\usepgfplotslibrary{dateplot}
\usepackage{xspace}
\usepackage{pbox}

% graphics path
\graphicspath{{./images/}}

% a few macros
\newcommand{\bi}{\begin{itemize}}
\newcommand{\ei}{\end{itemize}}
\newcommand{\ig}{\includegraphics}

% title info
\title{Análisis de complejidad computacional --- Complejidad de tiempo}
\author{Miguel Ortiz}
\institute{Abril 2023 - Cochabamba, Bolivia}
\date{\textbf{Programación competitiva para ICPC}}

% Tikz
\usepackage{tikz}
\usetikzlibrary{arrows,shapes}

% Minted
\usepackage{minted}
\usemintedstyle{manni}
\newminted{cpp}{fontsize=\footnotesize}

% Graph styles
\tikzstyle{vertex}=[circle,fill=black!50,minimum size=15pt,inner sep=0pt, font=\small]
\tikzstyle{selected vertex} = [vertex, fill=red!24]
\tikzstyle{edge} = [draw,thick,-]
\tikzstyle{dedge} = [draw,thick,->]
\tikzstyle{weight} = [font=\scriptsize,pos=0.5]
\tikzstyle{selected edge} = [draw,line width=2pt,-,red!50]
\tikzstyle{ignored edge} = [draw,line width=5pt,-,black!20]


\begin{document}
\maketitle

\begin{frame}{Introducción}
    \bi
        \item El análisis de complejidad mide la cantidad de recursos que utiliza un algoritmo
        \item La eficiencia de algoritmos es \textit{muy} importante en competencias de programación
    \ei
\end{frame}

\begin{frame}{Complejidad de tiempo}
    \bi
        \item Para motivos prácticos, estima el tiempo que demora un algoritmo en función al tamaño de la entrada
        \item Regla general: $10^8$ operaciones por segundo
    \ei
\end{frame}

\section{Notación Big O}

\begin{frame}{Notación Big O}
    Sea $T(n)$ una función; dada otra función $f(n)$, se dice que $T(n)$ es 
    $O(f(n))$ si existen constantes $c > 0$ y $n_0 \geq 0$, tales que para todo $n \geq n_0$ 
    se tenga $T(n) \leq c \cdot f(n)$
\end{frame}

\begin{frame}{Notación Big O}
  \bi
    \item $O(...) \rightarrow$ "$O$ de ..." 
    \item ... $\rightarrow$ expresión matemática. Ej.: $n, n^2, n+m, n2^n$
    \item <2-> Digamos $O(n^2)$. Reemplazamos $n$ por su valor máximo en el problema
    \item <2-> $n \leq 10^5 \rightarrow n^2 = 10^{10} \leftarrow$ cantidad aproximada de operaciones 
    \item <3-> ¿Es eficiente?
  \ei
\end{frame}

\begin{frame}{Notación Big O}
  \bi
    \item $X \leftarrow$ Cantidad aproximada de operaciones
    \vspace{10pt}
    \item $X \leq 10^7 \rightarrow$ Probablemente sea \textcolor{teal}{\underline{rápido}}
    \vspace{10pt}
    \item $X \approx 10^8 \rightarrow$ Zona de peligro
    \vspace{10pt}
    \item $X \geq 10^9 \rightarrow$ Probablemente sea muy \textcolor{red}{\underline{lento}}
  \ei
\end{frame}

\begin{frame}[fragile]{Notación Big O -- Operaciones simples}
  \begin{minted}[]{cpp}
    int n;
    cin >> n;
    int respuesta = n+1;
    cout << respuesta << '\n';
  \end{minted}
  \bi
    \item Cantidad constante de operaciones $\rightarrow O(1)$
    \item Si $n = 10^5$, el código de arriba sigue siendo $O(1)$
  \ei
\end{frame}

\begin{frame}[fragile]{Notación Big O -- Ciclos}
  \begin{minted}[]{cpp}
    for (int i = 1; i <= n; ++i) {
      // Código O(1)
    }
  \end{minted}
  \bi
    \item \textit{Código} se realiza $n$ veces $\rightarrow O(n)$
  \ei
\end{frame}

\begin{frame}[fragile]{Notación Big O -- Ciclos}
  \begin{minted}[]{cpp}
    for (int i = 1; i <= n; ++i) {
      for (int j = 1; j <= n; ++j) {
        // Código O(1)
      }
    }
  \end{minted}
  \bi
    \item \textit{Código} se realiza $n \times n$ veces $\rightarrow O(n^2)$
    \item <2-> Regla general: si hay $k$ ciclos anidados, la complejidad es $O(n^k)$
  \ei
\end{frame}

\begin{frame}[fragile]{Notación Big O -- Orden de magnitud}
  \begin{minted}[]{cpp}
    for (int i = 1; i <= 3*n; ++i) {
      // Código O(1)
    }
  \end{minted}
  \bi
    \item Realiza $3n$ operaciones
    \item <2-> $O(n)$
  \ei
\end{frame}

\begin{frame}[fragile]{Notación Big O -- Orden de magnitud}
  \begin{minted}[]{cpp}
    for (int i = 1; i <= n+5; ++i) {
      // Código O(1)
    }
  \end{minted}
  \bi
    \item Realiza $n+5$ operaciones
    \item <2-> $O(n)$
  \ei
\end{frame}

\begin{frame}[fragile]{Notación Big O -- Orden de magnitud}
  \begin{minted}[]{cpp}
    for (int i = 1; i <= n; i += 2) {
      // Código O(1)
    }
  \end{minted}
  \bi
    \item Realiza $n/2$ operaciones
    \item <2-> $O(n)$
  \ei
\end{frame}

\begin{frame}[fragile]{Notación Big O -- Fases}
  \begin{minted}[]{cpp}
    for (int i = 1; i <= n; ++i) {
      // Código O(1)
    }
    for (int i = 1; i <= n; ++i) {
      for (int j = 1; j <= n; ++j) {
        // Código O(1)
      }
    }
    for (int i = 1; i <= n; ++i) {
      // Código O(1)
    }
  \end{minted}
  \bi
    \item $O(n + n^2 + n)$
    \item <2-> $O(n^2)$
  \ei
\end{frame}

\begin{frame}[fragile]{Notación Big O -- Más de una variable}
  \begin{minted}[]{cpp}
    for (int i = 1; i <= n; ++i) {
      for (int j = 1; j <= m; ++j) {
        // Código O(1)
      }
    }
  \end{minted}
  \bi
    \item Realiza $n \times m$ operaciones $\rightarrow O(nm)$
  \ei
\end{frame}

\begin{frame}{Notación Big O -- Recomendación}
  \begin{center}
    Siempre calculen la complejidad de sus soluciones
  \end{center}
\end{frame}

\section{Ejercicios}

\begin{frame}[fragile]{Ejercicios}
  \begin{minted}[]{cpp}
    int a = 1, b = 2;
    int c = a + b;
  \end{minted}
  \bi
    \item Complejidad: \onslide<2-> $O(1)$
  \ei
\end{frame}

\begin{frame}[fragile]{Ejercicios}
  \begin{minted}[]{cpp}
    int n;
    cin >> n
    int res = 0;
    for (int i = 0; i < n; ++i) {
      res = res + i;
    }
    cout << res << endl;
  \end{minted}
  \bi
    \item Complejidad: \onslide<2-> $O(n)$
  \ei
\end{frame}

\begin{frame}[fragile]{Ejercicios}
  \begin{minted}[]{cpp}
    int n;
    cin >> n
    int res = 0;
    for (int i = 0; i < n; ++i) {
      for (int j = 0; j < n; ++j) {
        res = res + i*j;
      }
    }
    cout << res << endl;
  \end{minted}
  \bi
    \item Complejidad: \onslide<2-> $O(n^2)$
  \ei
\end{frame}

\begin{frame}[fragile]{Ejercicios}
  \begin{minted}[]{cpp}
    int n;
    cin >> n
    int res = 0;
    for (int i = 0; i < 3*n; ++i) {
      res = res + i%3;
    }
    cout << res << endl;
  \end{minted}
  \bi
    \item Complejidad: \onslide<2-> $O(n)$
  \ei
\end{frame}

\begin{frame}[fragile]{Ejercicios}
  \begin{minted}[]{cpp}
    int n;
    cin >> n
    int res = 0;
    for (int i = 0; i < n; ++i) {
      for (int j = 0; j < n; ++j) {
        res = res - i*j;
      }
      for (int j = 0; j < n; ++j) {
        for (int k = 0; k < n; ++k) {
          res = res + i*(j - k);
        }
      }
    }
    cout << res << endl;
  \end{minted}
  \bi
    \item Complejidad: \onslide<2-> $O(n \times (n + n^2))$ \onslide<3-> $\rightarrow O(n^3)$
  \ei
\end{frame}

\begin{frame}[fragile]{Ejercicios}
  \begin{minted}[]{cpp}
    int n;
    cin >> n
    int res = 0;
    for (int i = 0; i < n; ++i) {
      for (int j = i; j < n; ++j) {
        res++;
      }
    }
    cout << res << endl;
  \end{minted}
  \bi
    \item Complejidad: \onslide<2-> $O(\frac{n(n+1)}{2})$ \onslide<3-> $\rightarrow O(n^2)$
  \ei
\end{frame}

\begin{frame}[fragile]{Ejercicios}
  \begin{minted}[]{cpp}
    vector<int> v;
    v.push_back(2023);
  \end{minted}
  \bi
    \item Complejidad: \onslide<2-> $O(1)$
  \ei
\end{frame}

\begin{frame}[fragile]{Ejercicios}
  \begin{minted}[]{cpp}
    set<int> s;
    s.insert(2023);
  \end{minted}
  \bi
    \item Complejidad: \onslide<2-> $O(\log_2 n) \rightarrow O(\log n)$
  \ei
\end{frame}

\begin{frame}[fragile]{Ejercicios}
  \begin{minted}[]{cpp}
    int n;
    cin >> n;
    set<int> s;
    for (int i = 0; i < n; ++i) {
      s.insert((2*i + i/3) % n);
    }
    cout << s.size() << endl;
  \end{minted}
  \bi
    \item Complejidad: \onslide<2-> $O(n \log n)$
  \ei
\end{frame}

\end{document}
